\subsection{Kiến thức trọng tâm}

%\subsubsection{Mệnh đề, mệnh đề chứa biến}
%
%\paragraph{Mệnh đề}
%
%Mệnh đề là một khẳng định hoặc \textit{đúng} hoặc \textit{sai}. 
%\begin{itemize}
%\item Một mệnh đề không thể vừa \textit{đúng} vừa \textit{sai}.
%\item Một khẳng định đúng được gọi là một mệnh đề đúng.
%\item Một khẳng định sai được gọi là một mệnh đề sai.
%\end{itemize}
%
%
%\begin{note}
%\begin{enumerate}[-]
%\item Các câu hỏi, câu cảm thán, câu mệnh lệnh không phải là mệnh đề.
%\item Một khẳng định chưa xác định được đúng hay sai nhưng chắc chắn nó chỉ hoặc đúng hoặc sai thì khẳng định đó cũng là một mệnh đề. Ví dụ: "Có sự sống ngoài Trái Đất" là một mệnh đề.
%\item Những câu không có tính chất đúng hoặc sai thì không là mệnh đề. Ví dụ: "$0,005$ là một số rất bé" không là một mệnh đề.
%\end{enumerate}
%\end{note}
%
%\begin{noteout}
%Một mệnh đề mà nội dung của nó liên quan đến toán học gọi là \textbf{mệnh đề toán học}.
%\end{noteout}
%
%\paragraph{Mệnh đề chứa biến}
%
%Ví dụ: Xét câu {\lq\lq}$n$ chia hết cho $3${\rq\rq} ($n$ là số tự nhiên). Với mỗi giá trị của $n$, ta xét được tính đúng, sai của câu trên. Câu phát biểu như vậy được gọi là \textbf{mệnh đề chứa biến}.
%
%\subsubsection{Mệnh đề phủ định}
%
%Cho mệnh đề $P$. Mệnh đề "không phải $P$" được gọi là \textbf{mệnh đề phủ định} của $P$, kí hiệu là $\overline{P}$.
%
%\begin{note}
%Mệnh đề $P$ đúng thì $\overline{P}$ sai và ngược lại.
%\end{note}
%
%\subsubsection{Mệnh đề kéo theo. Hai mệnh đề tương đương}
%
%Cho hai mệnh đề $P$ và $Q$.
%\begin{itemize}
%\item Mệnh đề "Nếu $P$ thì $Q$" được gọi là \textbf{mệnh đề kéo theo}, kí hiệu là $P \Rightarrow Q$.
%\item Xét mệnh đề $P \Rightarrow Q$, ta nói:
%\begin{enumEX}[-]{2}
%\item $Q$ là điều kiện cần để có $P$.
%\item $P$ là điều kiện đủ để có $Q$.
%\end{enumEX}
%\item Mệnh đề $Q \Rightarrow P$ là \textbf{mệnh đề đảo} của của mệnh đề $P \Rightarrow Q$.
%\item Nếu cả hai mệnh đề $P\Rightarrow Q$ và $Q\Rightarrow P$ đều đúng thì ta nói $P$ và $Q$ là \textbf{hai mệnh đề tương đương}, kí hiệu là $P \Leftrightarrow Q$.
%\end{itemize}
%
%\subsubsection{Mệnh đề chứa kí hiệu $\boldsymbol{\forall}$, $\boldsymbol{\exists}$}
%
%\begin{itemize}
%\item $\forall$: chỉ "tất cả" các đối tượng đang xét (đọc là với mọi).
%\item $\exists$: chỉ rằng "có ít nhất một" trong các đối tượng đang xét (đọc là tồn tại).
%\end{itemize}
%
%\textbf{Ví dụ 3.} Các mệnh đề sau có sử dụng kí hiệu $\forall$, $\exists$:
%\begin{enumerate}[i)]
%\item $\forall n\in \mathbb{N}, n$ là số chẵn. (đọc là: "Với mọi số tự nhiên $n$ thì $n$ là số chẵn")
%\item $\exists n\in \mathbb{N}^*, n$ là số chẵn. (đọc là: "Tồn tại ít nhất một số tự nhiên $n$ khác $0$ để $n$ là số chẵn")
%\end{enumerate}

\subsection{Các dạng bài tập và phương pháp giải}

\begin{dang}{Công thức cộng}
	\begin{itemize}
		\item[$\bullet$] $\sin (\alpha + \beta)=\sin \alpha\cos \beta+ \sin \beta\cos a$.
		\item[$\bullet$] $\sin (\alpha - \beta)=\sin \alpha\cos \beta-\sin \beta\cos a$.
		\item[$\bullet$] $\cos (\alpha + \beta)=\cos \alpha\cos \beta - \sin \alpha\sin \beta$.
		\item[$\bullet$] $\cos (\alpha - \beta)=\cos \alpha\cos \beta +\sin \alpha\sin \beta$.
		\item[$\bullet$] $\tan (\alpha + \beta)=\dfrac{\tan \alpha + \tan \beta}{1-\tan a\tan \beta}$.
		\item[$\bullet$] $\tan (\alpha - \beta)=\dfrac{\tan \alpha - \tan \beta}{1+\tan \alpha\tan \beta}$.
	\end{itemize}
\end{dang}

\VDMH

\begin{vd}
Cho $\cos a=-\dfrac{12}{13}$ và $\pi<a<\dfrac{3\pi}{2}$.
Tính $\sin\left(\dfrac{\pi}{3}-a\right)$.
\loigiai
{
Vì $\pi<a<\dfrac{3\pi}{2}$ nên $\sin a<0$.\\
$\sin^{2}a=1-\cos^{2}a=1-\left(-\dfrac{12}{13}\right)^2=\dfrac{25}{169} \Rightarrow \sin a=-\dfrac{5}{13}$.\\
Do đó \\
$\sin\left(\dfrac{\pi}{3}-a\right)=\sin\dfrac{\pi}{3}\cdot\cos a-\cos\dfrac{\pi}{3}\cdot\sin a=\dfrac{\sqrt{3}}{2}\cdot\left(-\dfrac{12}{13}\right)-\dfrac{1}{2}\cdot \left(-\dfrac{5}{13}\right)=\dfrac{5-12\sqrt{3}}{26}.$
}
\end{vd}

\BTVD

\Opensolutionfile{ansbook}[ans/ansbook0D-1-1.1]



\Closesolutionfile{ansbook}
\HDG
\input{ans/ansbook0D-1-1.1}

